\documentclass[a4paper,11pt]{article}

\newcommand{\authorinfo}{Paul Bienkowski, Konstantin Kobs}
\newcommand{\titleinfo}{Robotics Assignment \#01}

% PREAMBLE ===============================================================

\usepackage[german,ngerman]{babel}
\usepackage[utf8]{inputenc}
\usepackage[T1]{fontenc}
\usepackage[top=1.3in, bottom=1in, left=1.0in, right=0.6in]{geometry}
\usepackage{lmodern}
\usepackage{amssymb}
\usepackage{mathtools}
\usepackage{amsmath}
\usepackage{enumerate}
\usepackage{pgfplots}
\usepackage{breqn}
\usepackage{tikz}
\usepackage{fancyhdr}
\usepackage{multicol}
\usepackage{url}

\usetikzlibrary{calc}
\usetikzlibrary{patterns}

\author{\authorinfo}
\title{\titleinfo}
\date{\today}

\pagestyle{fancy}
\fancyhf{}
\fancyhead[L]{\authorinfo}
\fancyhead[R]{\titleinfo}
\fancyfoot[C]{\thepage}

\begin{document}
\maketitle
\begin {enumerate}
\item[\textbf{Task 1.1.}]
	\textbf{Note:} All the code for calculating the results can be found at \url{http://hiernocheinfügen.de}.\\
	The given points are $A = (5, -5, 0), B = (5, 5, 0), C = (-5, 5, 0), D = (-5, -5, 0), E = (0, 0, 20)$
    \begin{enumerate}
        \item[1)] In this case we use Euler-angles, so we need to left multiply the given transformation matrices. We get the following transformation matrix:
		$$M \approx \begin{bmatrix}
			0.78914913099243 & -0.61237243569579 & 0.65973960844117 & 0\\
-0.78914913099243 & 0.61237243569579 & -0.65973960844117 & 0\\
0.43301270189222 & 0.5 & 0.75 & 0\\
0 & 0 & 0	 & 1
		\end{bmatrix}$$
		After multiplying it with the given points, we get
		\begin{align*}
			A' &\approx (7.0076078334411, -7.0076078334411, -0.3349364905389),\\
			B' &\approx (0.88388347648318, -0.88388347648318, 4.6650635094611),\\
			C' &\approx (-7.0076078334411, 7.0076078334411, 0.3349364905389),\\
			D' &\approx (-0.88388347648318, 0.88388347648318, -4.6650635094611),\\
			E' &\approx (19.792188253235, -19.792188253235, 22.5)
		\end{align*}

        
        \item[2)] In this case we use Gimbal-angles, we need to right multiply the given transformation matrices. We get the following transformation matrix:
		$$M' \approx \begin{bmatrix}
			0.43559574039916 & -0.43559574039916 & 0.43301270189222 & 0\\
-0.61237243569579 & 0.61237243569579 & -0.5 & 0\\
0.047367172745376 & -0.047367172745376 & 0.75 & 0\\
0 & 0 & 0 & 1
		\end{bmatrix}$$
		After multiplying it with the given points, we get
		\begin{align*}
			A'' &\approx (4.3559574039916, -6.1237243569579, 0.47367172745376),\\
			B'' &\approx (4.4408920985006 \cdot 10^{-16}, 4.4408920985006  \cdot 10^{-16}, 5.5511151231258 \cdot 10^{-16}),\\
			C'' &\approx (-4.3559574039916, 6.1237243569579, -0.47367172745376),\\
			D'' &\approx (-4.4408920985006 \cdot 10^{-16}, -4.4408920985006 \cdot 10^{-16}, -5.5511151231258 \cdot 10^{-16}),\\
			E'' &\approx (12.990381056767, -15, 22.5)
		\end{align*}
    \end{enumerate}

\item[\textbf{Task 1.2.}]

    \begin{enumerate}
        \item[1)]
        \item[2)]
    \end{enumerate}

\item[\textbf{Task 1.3.}]

    \begin{enumerate}
        \item[1)]
        \item[2)] If we rotate an object around one axis, the coordinate system changes in the other two axis, but not in the rotated one. Rotating again around the same axis is senseless, because we could have achieved that by the first rotation. So we have the following number of rotation sequences:
        $$n_{\text{sequences}} = 3 \cdot 2 \cdot 2 = 12$$
    \end{enumerate}

\end {enumerate}
\end{document}
