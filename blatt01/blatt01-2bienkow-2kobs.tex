\documentclass[a4paper,11pt]{article}

\newcommand{\authorinfo}{Paul Bienkowski, Konstantin Kobs}
\newcommand{\titleinfo}{Robotics Assignment \#01}

% PREAMBLE ===============================================================

\usepackage[german,ngerman]{babel}
\usepackage[utf8]{inputenc}
\usepackage[T1]{fontenc}
\usepackage[top=1.3in, bottom=1in, left=1.0in, right=0.6in]{geometry}
\usepackage{lmodern}
\usepackage{amssymb}
\usepackage{mathtools}
\usepackage{amsmath}
\usepackage{enumerate}
\usepackage{pgfplots}
\usepackage{breqn}
\usepackage{tikz}
\usepackage{fancyhdr}
\usepackage{multicol}
\usepackage{url}
\usepackage{gensymb}

\usetikzlibrary{calc}
\usetikzlibrary{patterns}

\author{\authorinfo}
\title{\titleinfo}
\date{\today}

\pagestyle{fancy}
\fancyhf{}
\fancyhead[L]{\authorinfo}
\fancyhead[R]{\titleinfo}
\fancyfoot[C]{\thepage}

\newcommand{\V}[1]{\begin{pmatrix}#1\end{pmatrix}}

\begin{document}
\maketitle
\begin {enumerate}
\item[\textbf{Task 1.1.}]

    \textbf{Note:} All the code for calculating the results can be found at \url{http://tinyurl.com/qbmvazo}.

    Assuming a right handed coordinate system with the z-axis going upwards
    towards point E, we have, considering the limitations on the edges being
    parallel to certain axes, a few permutations for the positions of the
    vertices A through D. We chose the following base positions:

    $$
    A = \V{-5\\-5\\0},
    B = \V{5\\-5\\0},
    C = \V{5\\5\\0},
    D = \V{-5\\5\\0},
    E = \V{0\\0\\30}
    $$

    We have 3 different transformation (rotation) matrices for the three steps:

    $$
        R_1 \approx \begin{bmatrix}
            0.7071& -0.7071 & 0 & 0\\
            0.7071& -0.7071 & 0 & 0\\
            0 & 0 & 1 & 0\\
            0 & 0 & 0 & 1
        \end{bmatrix},
        R_2 \approx \begin{bmatrix}
            1 & 0 & 0 & 0 \\
            0 & 0.8660 & -0.5 & 0 \\
            0 & 0.5 & 0.8660 & 0 \\
            0 & 0 & 0 & 1
        \end{bmatrix},
        R_3 \approx \begin{bmatrix}
            0.8660& 0 & -0.5 & 0\\
            0 & 1 & 0 & 0\\
            0.5 & 0 & 1 & 0.8660\\
            0 & 0 & 0 & 1
        \end{bmatrix}
    $$

>>>>>>> Stashed changes
    \begin{enumerate}
        \item[1)]

            In this case we use Euler angles, so we need to left multiply the
            given transformation matrices. We get the following transformation
            matrix/transformed vertices:

        $$M_1 = R_3 R_2 R_1 \approx \begin{bmatrix}
            0.4356 & -0.7891 & -0.4330 & 0 \\
            0.6124 &  0.6124 & -0.5 & 0 \\
            0.6597 & -0.0474 &  0.75 & 0 \\
            0 & 0 & 0 & 1
        \end{bmatrix}$$

        $$
            A_1 \approx \begin{pmatrix}1.7678 \\ -6.1237 \\ -3.0619\end{pmatrix}
            B_1 \approx \begin{pmatrix}6.1237 \\ -0.0000 \\ 3.5355\end{pmatrix}
            C_1 \approx \begin{pmatrix}-1.7678 \\ 6.1237 \\ 3.0619\end{pmatrix}
            D_1 \approx \begin{pmatrix}-6.1237 \\ 0.0000 \\ -3.5355\end{pmatrix}
            E_1 \approx \begin{pmatrix}-12.9904 \\ -15.0000 \\ 22.5000\end{pmatrix}
        $$


        \item[2)]

            In this case we use Gimbal angles, we need to right multiply the
            given transformation matrices. We get the following transformation
            matrix/transformed vertices:

        $$M_2 = R_1 R_2 R_3 \approx \begin{bmatrix}
            0.7891 & -0.6124 & -0.0474 & 0.0000\\
            0.4356 & 0.6124 & -0.6597 & 0.0000\\
            0.4330 & 0.5000 & 0.7500 & 0.0000\\
            0.0000 & 0.0000 & 0.0000 & 1.0000
        \end{bmatrix}$$

        $$
            A_1 \approx \begin{pmatrix}-0.8839 \\ -5.2398 \\ -4.6651\end{pmatrix},
            B_1 \approx \begin{pmatrix}7.0076 \\ -0.8839 \\ -0.3349\end{pmatrix},
            C_1 \approx \begin{pmatrix}0.8839 \\ 5.2398 \\ 4.6651\end{pmatrix},
            D_1 \approx \begin{pmatrix}-7.0076 \\ 0.8839 \\ 0.3349\end{pmatrix},
            E_1 \approx \begin{pmatrix}-1.4210 \\ -19.7922 \\ 22.5000\end{pmatrix}
        $$

    \end{enumerate}

\item[\textbf{Task 1.2.}]

    \begin{enumerate}
        \item[1)] To calculate the homogeneous transformation from A to C, or short $^{A}T_{C}$, we need to right multiply $^{A}T_{B}$ and $^{B}T_{C}$. Intuitively we transform all points in A to B and then to C. That's why we need to right multiply the matrices. So the transformation $^{A}T_{C}$ is unambiguously defined.
        \item[2)]
    \end{enumerate}

\item[\textbf{Task 1.3.}]

    \begin{enumerate}
        \item[1)]

            Examples for $(\phi, \theta, \psi)$ are:

            \begin{enumerate}
                \item $(90\degree, -20\degree, 30\degree)$ in $ZY'X''$ means a $90\degree$ rotation around the z-axis, a rotation by $-20\degree$ around the newly defined y-axis and a $30\degree$ rotation around the new frame's x-axis.
                \item $(-15\degree, 100\degree, 12\degree)$ in $XZ'Y''$ means a $-15\degree$ rotation around the x-axis, a rotation by $100\degree$ around the newly defined z-axis and a $12\degree$ rotation around the new frame's y-axis.
                \item $(30\degree, 10\degree, -50\degree)$ in $YX'Y''$ means a $30\degree$ rotation around the y-axis, a rotation by $10\degree$ around the newly defined x-axis and a $-50\degree$ rotation around the new frame's y-axis.
            \end{enumerate}

        \item[2)]

            If we rotate an object around one axis, the coordinate system
            changes in the other two axes, but not in the rotated one. Rotating
            again around the same axis is senseless, because we could have
            achieved that with the first rotation. Hence, we have 3 choices for
            the first axis, and 2 choices for each remaining axis, resulting in
            the following number of rotation sequences:

            $$n_{\text{sequences}} = 3 \cdot 2 \cdot 2 = 12$$

    \end{enumerate}

\end {enumerate}
\end{document}
